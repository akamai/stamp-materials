\documentclass[a4paper]{tufte-book}
\usepackage{booktabs}
\usepackage{tabularx}
\usepackage{longtable} 
\usepackage{lscape}
\usepackage{colortbl}

\geometry{
  left=.5in,
  right=.5in,
  top=.5in,
  bottom=.5in
}

\begin{document}

\newtheorem{example}{Ex}

\begin{landscape}
\advance\vsize6cm
\csname @colroom\endcsname=\vsize
\textheight=\vsize
\csname @colht\endcsname=\vsize

\setlength{\parindent}{0em}
\setlength{\parskip}{.75em}

\begin{multicols}{2}
[ \section{Identifying Hazards}]

\newthought{Definition}

\textbf{Hazards} are system conditions that, in combination with environmental conditions outside our control, can result in a loss.

Our task is to \textbf{write a list} of hazards, staying broadly general and covering all the losses.

For each of the \textbf{losses} defined earlier, we'll identify one or more \textbf{hazards}. 



\newthought{Thermostat Example}

\begin{tabular}{|p{.5cm}|p{8.5cm}|}
\hline
&\textsc{Losses}\\
\hline
L1 & Room gets too cold (2 or more degrees below target)\\
\hline
L2 & Room gets too hot (2 or more degrees above target)\\
\hline
L3 & Damage to facilities, property, or the heating equipment itself\\
\hline
L4 & Waste of fuel\\
\hline
L5 & Physical harm to humans or pets\\
\hline
\end{tabular}
\vspace{1em}


\begin{tabular}{|p{.75cm}|p{6.75cm}|p{2cm}|}
\hline
&\textsc{Hazards}&\textsc{Losses}\\
\hline
H1&HEAT ON when room is already warm (2 or more degrees above target)&L2, L4, L3\\
\hline
H1.1&Heater can't turn off&L2, L4\\
\hline
H2&HEAT OFF when room is already cold (2 or more degrees below target)&L1, L3, L4\\
\hline
H2.1&Heater can't turn on&L1\\
\hline
H3&Short cycles of HEAT ON and HEAT OFF&L4, L3?\\
\hline
\end{tabular}
\vspace{1em}

This list of hazards is not yet complete; we have not identified hazards for \emph{L5: Physical harm to humans or pets}, and we haven't expressed much about \emph{L3: Damage to facilities, property, or the heating equipment itself}.


%It may be tempting to start thinking about the following:


%\begin{tabular}{|l|l|l|}
%\hline
%H6& Thermostat-measured temp differs from actual room temp by >X degrees.&L1-L4\\
%\hline
%H7& Substantial lag between the change in actual temp and the change in measurement.&L1-L4\\
%\hline
%H8& Substantial lag between signaling HEAT ON or HEAT OFF and furnace turning on or off.&L1-L4\\
%\hline
%\end{tabular}

\columnbreak

\newthought{Desired Qualities} 
\begin{itemize}
\setlength{\itemsep}{0pt}
\setlength{\parskip}{.25em}
\item Concise --- We want a relatively short list
\item General --- We don't want to prematurely narrow our focus. 
\item Good coverage --- For any accident we can think up, we want it to be described by at least one of the hazards on this list.
\item Non-redundant --- Overlap between hazards is ok, but if one loss is entirely a subset of another, perhaps consider consolidating them.
\item Under our control --- For them to be useful in guiding our actions, hazards should identify conditions we can actually do something about. Things outside our control (like weather, meteors, or the popularity of particular websites) are environmental conditions.
\item Relevant --- They should be associated with the losses in a meaningful way. 

Perhaps list what environmental condition would result in the loss.
\end{itemize}  
 
\newthought{Strategic Approaches} 

Ask "what is \emph{risky but tolerable}?" vs. "what is \emph{unacceptable}?" to distinguish from losses--- What is a priority?

\newthought{Relationship to other concepts}

The relationship between \textbf{losses} and \textbf{hazards} lets us \emph{prioritize} our safety efforts, focusing on preventing the system states that are relevant to producing these accidents--- we don't need to examine every combination of system states.

\end{multicols}
\end{landscape}
\end{document}
